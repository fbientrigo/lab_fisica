\documentclass[a4paper,twocolumn,10pt]{article}
\usepackage[spanish]{babel}
\usepackage[T1]{fontenc}
\usepackage[utf8]{inputenc}
\spanishdecimal{.}
\usepackage{lmodern}
\usepackage[a4paper]{geometry}
\usepackage{graphicx}
\usepackage{flushend}
\usepackage{wallpaper}
\usepackage{amsmath}
\usepackage{float}
\usepackage{colortbl}

\begin{document}

\title{Experimentos con microondas}
\author{ \\Aldo Aliaga, Benjamín Yapur, Fabian Trigo \\ \textit{Departamento de Física y Astronomía, Universidad de Valparaiso}}
\twocolumn[
  \begin{@twocolumnfalse}
    \maketitle
    \begin{abstract}
    El objetivo de este experimento es estudiar el comportamiento de las microondas y comprobar que cumplen las propiedades típicas de las ondas como la reflexión, difracción e interferencia mediante el uso de un equipo generador (receptor) de microondas, el cual permitió emitir (medir) las ondas luego de hacerlas pasar por distintas configuraciones. Llegando a una coherencia entre lo observado experimentalmente y lo esperado teóricamente sobre los fenómenos que sufren estas ondas.
    \end{abstract}
  \end{@twocolumnfalse}\bigskip]

\vspace{2cm}

\section{Introducción}

Las microondas son un tipo de radiación electromagnética con longitudes de onda que van desde un metro a un milímetro, correspondientes al rango de frecuencias entre $300[MHz]$ y $300[GHz]$. Su naturaleza ondulatoria le otorga las propiedades de reflexión, refracción, difracción e interferencia. 

Se puede conocer la frecuencia de las microondas emitidas relacionando su longitud de onda con la velocidad de propagación de las ondas en el medio:
\begin{equation}
    f=\frac{c}{\lambda}
\end{equation}


\section{Montaje}

Primeramente, se realiza el montaje del sistema de guías, conformado de dos rieles y un plato graduado con una montura para el porta placas.

Se cuenta con una antena de bocina emisora y una receptora. Estas se orientarán en el sistema de guías de acuerdo a cada experimento.

También se utiliza una fuente de poder en la cual se puede regular la amplificación de la señal, conectar las bocinas y conectar un voltímetro para medir la intensidad de la señal.

A continuación se detalla el montaje para cada experiencia:

\subsection*{1 Propagación Lineal del Microondas}
Detector frente al emisor
\begin{figure}[H]
    \centering
    \includegraphics{MO_montaje/1.png}
    \caption{Montaje del experimento de propagacion lineal del microondas}
    \label{fig:proplineal}
\end{figure}

\subsection*{2 Poder de Penetración}

\begin{figure}[H]
    \centering
    \includegraphics{MO_montaje/2.png}
    \caption{Montaje del experimento del poder de penetracion}
    \label{fig:montpen}
\end{figure}

\subsection*{3 Apantallamiento y Absorción}
Utilizando dos materiales distintos se mide el cambio de intensidad con cada uno, uno es amde
\begin{figure}[H]
    \centering
    \includegraphics{MO_montaje/3.png}
    \caption{Montaje del experimento que comprueba apantallamiento y absorción}
    \label{fig:montapant}
\end{figure}

\subsection*{4 Reflexión}

\begin{figure}[H]
    \centering
    \includegraphics{MO_montaje/4.png}
    \caption{Experimento de reflexión}
    \label{fig:montreflex}
\end{figure}

Con emisor/receptor en un ángulo de 90 grados a la misma distancia de 200 milímetros del centro, se controla la normal de la pantalla conductora, la cual funciona como reflector.

\subsection*{5. Onda Estacionaria, Longitud de Onda}
\begin{figure}[H]
    \centering
    \includegraphics{MO_montaje/5.png}
    \caption{Montaje para medir longitud de onda.}
    \label{fig:my_label}
\end{figure}

\subsection*{6. Refracción}
Este experimento fue omitido por recomendación del profesor y solo sera listado su construcción, pero no aparecerá en el análisis
\begin{figure}[H]
    \centering
    \includegraphics{MO_montaje/6.png}
    \caption{Refracción con material}
    \label{fig:montrefr}
\end{figure}

\subsection*{7. Principio de Hyugens}
\begin{figure}[H]
    \centering
    \includegraphics[scale=.7]{MO_montaje/7.png}
    \caption{Montaje experimento principio de Hyugens}
    \label{fig:hyu}
\end{figure}

\subsection*{8. Difracción}
Utilizando un sensor se mide la intensidad a través de dos rendijas
\begin{figure}[H]
    \centering
    \includegraphics{MO_montaje/8.png}
    \caption{Experimento de difracción}
    \label{fig:montdif}
\end{figure}

\subsection*{9. Interferencia}
Se colocan un rendija simple y una rendija doble en el soporte portaplacas, con la bocina emisora a 12 (cm) de la placa y la sonda receptora unos 6 (cm) de la de la placa. La sonda es desplazada en línea recta por un eje paralelo a la placa.
\begin{figure}[H]
    \centering
    \includegraphics[scale=.7]{MO_montaje/9.png}
    \caption{Montaje experimento 9.}
    \label{fig:montinter}
\end{figure}

\subsection*{10. Polarización}
Se enclava la rendija de polarización en el portaplacas a 250 (mm) de las bocinas. Primero de forma vertical, luego horizontal y finalmente en diagonal. Se mide la intensidad de la señal para cada caso y también rotando las bocinas en 45 y 90 grados.
\begin{figure}[H]
    \centering
    \includegraphics[scale=.7]{MO_montaje/10.png}
    \caption{Montaje del experimento de polarización}
    \label{fig:montpol}
\end{figure}





\section{Análisis}
Gran parte de los experimentos son de carácter cualitativo, aquellos numéricos se proporcionara una tabla con los datos, el primer paso para analizar, fue apagar todo emisor y medir el voltaje del receptor, un voltaje base de $0.37 [V]$ considerado como el voltaje de ruido, por tanto cada vez que se tenga un voltaje cercano a este puede considerarse 0.

\subsection*{1 Propagación Lineal del Microondas}
Frente a frente tienen una intensidad de $3.16 [V]$, osea en $0$ grados, de aqui en adelante se listara en una tabla:
\begin{table}[H]
\centering
\caption{Intensidad vs el ángulo del receptor referente al emisor}
\begin{tabular}{rr}
\rowcolor[rgb]{0.753,0.753,0.753} \multicolumn{1}{l}{angulo [grad]} & \multicolumn{1}{l}{intensidad [V]}  \\
0                                                                   & 3.16                                \\
10                                                                  & 2.75                                \\
15                                                                  & 2.43                                \\
20                                                                  & 1.96                                \\
25                                                                  & 1.3                                 \\
30                                                                  & 0.75                                \\
35                                                                  & 0.5                                 \\
40                                                                  & 0.4                                
\end{tabular}
\end{table}

Se observa que decae conforme el ángulo se aleja, es decir, la onda se propaga de manera lineal en el medio.

\subsection*{2 Poder de Penetración}
El voltaje con la amplitud del aparato, en linea recta y sin nada en su camino, fue de: $4.87 [V]$
En el caso de tener un obstáculo de madera, el voltaje: $4.37 [V]$ osea se apantalla levemente. De esto se puede decir que las ondas logran atravesar la barrera disminuyendo su intensidad solo en un $10\%$.

\subsection*{3 Apantallamiento y Absorción}
El voltaje con la amplitud del aparato, en linea recta y sin nada en su camino, fue de: $4.87 [V]$
mientras que con un conductor en su camino: $0.39 [V]$, desciende al nivel del ruido del aparato, considerado como 0. Pudiendo así relacionar la conductividad de un material y su capacidad de absorber las ondas electromagnéticas.
\subsection*{4 Reflexión}
Con la normal de la pantalla metálica (reflector) en un ángulo de 60 grados relativo al 0 del receptor, el voltaje fue de $1.63 [V]$, hasta que a los 45 grados se obtuvo el máximo: $4.61 [V]$.
Sigue a la perfección la ley de reflexión para las ondas.

\subsection*{5. Onda Estacionaria, Longitud de Onda}
Configurando la amplitud del emisor a la mitad de la capacidad del generador para tener mas intensidad sin estropear la maquina, se obtienen los siguientes máximos:
\begin{table}[H]
\centering
\caption{Máximos registrados en un intervalo, para el calculo de $\lambda$}
\begin{tabular}{rr}
\rowcolor[rgb]{0.753,0.753,0.753} \multicolumn{1}{l}{Distancia [mm]} & \multicolumn{1}{l}{intensidad [V]}  \\
346                                                                  & 3.95                                \\
315                                                                  & 3.46                               
\end{tabular}
\end{table}

De esa manera la longitud de onda $\lambda$:
$$
\lambda = 346 - 315 = 31 [mm]
$$
Luego reemplazando para comparar con la frecuencia del aparato ($9.3 [GHz]$)
$$
f = \frac{c}{\lambda} [\frac{m/s}{m}] = 9.67 [GHz]
$$
Con tan solo un error de $9.67 - 9.3 [GHz] = 0.37 [GHz]$

\subsection*{7. Principio de Hyugens}
Se observo que el ángulo donde consigue llegar señal ahora es mayor, producto de que la rendija funciona como un nuevo origen de la onda. Queda así comprobado el principio de Hyugens.

\subsection*{8. Difracción}
Al mover el detector por detrás, se registra un incremento de intensidad cerca de los bordes, pero una vez en el centro (tras la pantalla conductora que no permite pasar las ondas), la intensidad decae bruscamente y vuelve a elevarse al aproximarse al borde sin aun mirar directamente al emisor. 
Quedo así vista la difraccion, pues la onda se comporta de manera similar a si volviera a generarse en los bordes.

\subsection*{9. Interferencia}
Al desplazar la sonda se registra la intensidad de la emisión con respecto al tiempo, mostrando mayor intensidad cuando la sonda está frente a la rendija. 
\begin{figure}[H]
    \centering
    \includegraphics[scale=.3]{PlotMOAnalisis/plot_una_rendija.png}
    \caption{Intensidad en función del tiempo a través de una rendija.}
    \label{fig:my_label}
\end{figure}

En la figura 11 se aprecia dicha relación para la rendija simple. De ella se puede apreciar un peak  central que representa el momento en que no hay interferencia entre la sonda y la bocina emisora, es decir, cuando está justo en frente de la rendija. Además se ven dos peaks menores a cada lado separados por dos valles del peak central, pero como hay interferencia en esos sectores, se asocian estos peaks a la superposición de las ondas difractadas por la rendija.
\begin{figure}[H]
    \centering
    \includegraphics[scale=.3]{PlotMOAnalisis/plot_doble_rendj.png}
    \caption{Intensidad en función del tiempo a través de una doble rendija.}
    \label{fig:my_label}
\end{figure}

En la figura 12 se aprecian tres peaks y dos laterales, siguiendo el mismo patrón gaussiano. Estos peaks y valles también se deben por la difracción producida por las rendijas y la interferencia de estas ondas difractadas.

Ambos patrones corresponden los de difracción e interferencia de las ondas electromagnéticas.



\subsection*{10. Polarización}
Primero se mide la el voltaje cuando el sistema está apagado y prendido sin \textit{ningún dispositivo que interfiera}, registrando 0.37[V] y 6.73[V] respectivamente. El hecho de que se reciba señal incluso cuando la emisión está apagada es debido al ruido de fondo. Mientras que si la bocina emisora es rotada 90° la intensidad disminuye a 0.77[V]. Este resultado indica que las microondas tienen una mayor amplitud en el eje vertical que en el horizontal.

Al tener el polarizador de forma \textit{vertical} sin rotar la bocina emisora se obtiene una intensidad de 0.55[V]. Mientras que al rotar la bocina emisora se obtiene 0.37[V]. De estos resultados es claro que el voltaje disminuye considerablemente en ambos casos, del primero de ellos se puede extraer que las microondas son horizontales cuando no se rota la bocina. Por ende la disminución del voltaje se debe netamente a la rejilla polarizadora. En el segundo resultado el polarizador terminó de suprimir las ondas (ahora verticales) que podían ser recibidas por la bocina receptora, anulando por completo la señal.

Posicionando el polarizador de forma \textit{horizontal} sin rotar la bocina emisora se mide 6.63[V]. Y al rotar la bocina se registra 0.37[V]. A partir del primer resultado se puede decir que el polarizador en esta posición solo disminuye en un $1.48\%$ el voltaje.
De este primer resultado las ondas verticales (predominantes) pasan por el polarizador, mientras que las horizontales son sustraídas por la placa.
El segundo resultado es el mismo caso que para el polarizador en forma vertical.

Luego se posicionó el \textit{polarizador en 45°}. Para esta configuración el voltaje se redujo a 6.21[V].

Dados estos resultados se comprueba que las ondas emitidas sufren polarización, confirmando también que son ondas transversales.


\section{Conclusión}
Se logró comprobar mediante todos los experimentos realizados que las microondas se comportan como predice la teoría acerca de las ondas transversales. Propagándose linealmente en el medio, siendo absorbidas en distintos porcentajes por materiales con distintos niveles de aislamiento, cumpliendo las leyes de reflexión, refracción y difracción. Además de cumplir con el principio de Hyugens, responder ante la interferencia superponiendo las ondas y sufrir polarización.


\section{Bibliografía}


\begin{itemize}
\item Arias, Arnaldo González (2001). ¿Qué es el magnetismo?. Universidad de Salamanca.
\end{itemize}


\end{document}