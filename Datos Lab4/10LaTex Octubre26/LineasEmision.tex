\documentclass[a4paper,twocolumn,10pt]{article}
\usepackage[spanish]{babel}
\usepackage[T1]{fontenc}
\usepackage[utf8]{inputenc}
\spanishdecimal{.}
\usepackage{lmodern}
\usepackage[a4paper]{geometry}
\usepackage{graphicx}
\usepackage{flushend}
\usepackage{wallpaper}
\usepackage{amsmath}
\usepackage{float}
\usepackage{colortbl}

\begin{document}

\title{Análisis Espectral: Determinación de las Longitudes de Onda de Distintos Elementos y Determinación de la Constante de Rydberg}
\author{ \\Aldo Aliaga, Benjamín Yapur, Fabian Trigo \\ \textit{Departamento de Física y Astronomía, Universidad de Valparaiso}}
\twocolumn[
  \begin{@twocolumnfalse}
    \maketitle
    \begin{abstract}
    Este experimento tiene dos objetivos principales: la determinación de las longitudes de onda y  la determinación de la constante de Rydberg. Para lograr esto, con el uso de un espectroscopio y una fuente de luz correspondiente a cada elemento, se midió el ángulo de difracción
    \end{abstract}
  \end{@twocolumnfalse}\bigskip]

\vspace{2cm}

\section{Introducción}

La ecuación para una red de difracción
\begin{equation}
\lambda = \frac{d sin(\theta)}{n}    
\end{equation}

\section{Montaje Experimental}
\subsection{Herramientas}
\begin{itemize} 
\item Telescopio
\item Lampara con distintas ampolletas con elementos a analizar
\item Lona para cubrir luz externa
\item Colimador
\item Lupa
\item Red de difracción
\item Escala con Vernier
\end{itemize}
\section{Análisis}

\section{Conclusión}
\section{Bibliografía}


\begin{itemize}
\item Thornton, S. T. \& Rex, A. (2022, 7 octubre). Modern Physics for Scientists and Engineers, 4th Edition (4.a ed.). Cengage Learning.
\end{itemize}


\end{document}

